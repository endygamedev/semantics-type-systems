\documentclass[landscape, 11pt]{report}

% Packages
\usepackage[landscape]{geometry}
\usepackage{amsmath}
\usepackage{xcolor}
\usepackage[utf8]{inputenc}
\usepackage[russian]{babel}
\usepackage{geometry}
\usepackage{graphicx}

% Options
\graphicspath{ {../figures/} {./figures/}}
\geometry{left=2.5cm,right=2.5cm,top=2.5cm,bottom=2.5cm}
\setlength\parindent{0pt}

% Title
\title{
\includegraphics[scale=0.07]{logo}\\
\vspace{0.5em}
Языки программирования. Семантика и система типов\\
\vspace{0.2em}
\Large Теоретическое задание. Тема 9
}
\author{Бронников Егор}
\date{}


\begin{document}

% Титул

\maketitle

% Задание

\vspace{-0.5cm}
\hrule
\vspace{0.5cm}

Следуя определениям императивных объектов с использованием открытой рекурсии, реализуйте подклассы \verb|SetCounter| с возможностью сохранять текущее состояние и откатываться к (одному из) прежних состояний. В своей реализации вы можете использовать абстрации-заглушки или ссылки-заглушки.

\vspace{0.5cm}
\hrule
\vspace{0.5cm}

% Задание 1

\textbf{Задание 1.} Реализуйте класс \verb|SingleBackupCounter| (функции \verb|singleBackupCounterClass| и \verb|newSingleBackupCounter|).

\begin{verbatim}
	SingleBackupCounter = {
	    get : Unit -> Nat,
	    set : Nat -> Unit,
	    inc : Unit -> Unit,
	    backup : Unit -> Unit,
	    restore : Unit -> Unit
	}
\end{verbatim}


\textit{Решение.}

\begin{verbatim}
	SingleBackupCounterRep = { x : Ref Nat, backup : Ref <nothing : Unit, just : T>}
\end{verbatim}

\begin{verbatim}
	SetCounter = {
	    get : Unit -> Nat,
	    set : Nat -> Unit,
	    inc : Unit -> Unit
	}
\end{verbatim}

\vfill

\footnotesize

\begin{center}
	\textit{Смотреть продолжение на следующей странице.}
\end{center}

\normalsize

\newpage

\begin{flalign*}
	&let \; singleBackupCounter = & \\
	&\quad \lambda \, rep : SingleBackupCounterRep \, .& \\
	&\qquad \lambda \, self : SingleBackupCounter .& \\
	&\qquad \quad let \; super = SetCounter \; rep \; self \; in& \\
	&\qquad \qquad \{ \, get = super .get,& \\
	&\qquad \qquad \; \, \, set = super .set,& \\
	&\qquad \qquad \; \, \, inc = super .inc,& \\
	&\qquad \qquad \; \, \, backup = \lambda \, \_ : Unit . \, rep.backup \, := \, <just = !rep.x>,& \\
	&\qquad \qquad \; \, \, restore = \lambda \, \_ : Unit . \, case \; !rep.backup \; of& \\
	&\hspace{5cm} <nothing = \_> \; => \; unit \; |& \\
	&\hspace{5cm} <just = x> \; => \; rep.x := x \; \}& \\
	&in& \\
	&\quad let \; newSingleBackupCounter =& \\
	&\qquad \lambda \_ : Unit .& \\
	&\qquad \quad let \; rep = \{ x = ref \; 0, \; backup = ref <nothing = unit> \} \; in& \\
	&\qquad \quad fix \; (singleBackupCounter \; rep)& \\
\end{flalign*}

\hrule
\vspace{0.5cm}

% Задание 2

\textbf{Задание 2.} Используйте списки, чтобы реализовать класс \verb|BackupCounter|, который хранит стек сохранённых состояний (представленный списком). Повторный вызов \verb|restore| должен быть поддержан (т.е. \verb|restore| сбрасывает историю сохранённых состояний).

\begin{verbatim}
	BackupCounter = {
	    get : Unit -> Nat,
	    set : Nat -> Unit,
	    inc : Unit -> Unit,
	    backup : Unit -> Unit,
	    restore : Nat -> Unit // аргумент - индекс на стеке
	}
\end{verbatim}

\vfill

\footnotesize

\begin{center}
	\textit{Смотреть продолжение на следующей странице.}
\end{center}

\normalsize

\newpage

\textit{Решение.}

\begin{verbatim}
	BackupCounterRep = { x : Ref Nat, backup : Ref List Nat }
\end{verbatim}

\vspace{-0.5cm}

\begin{flalign*}
	&letrec \; getByIndex = & \\
	&\quad \lambda \, list : List \, Nat \, .& \\
	&\quad \lambda \, index : Nat .& \\
	&\qquad if \; \, iszero \; index \; \, then \; head \; list \; \, else \; \, getByIndex \, (tail \; list) \, (pred \; index)&
\end{flalign*}

\vspace{-0.5cm}

\begin{flalign*}
	&let \; backupCounter = & \\
	&\quad \lambda \, rep : BackupCounterRep \, .& \\
	&\qquad \lambda \, self : BackupCounter .& \\
	&\qquad \quad let \; super = SetCounter \; rep \; self \; in& \\
	&\qquad \qquad \{ \, get = super .get,& \\
	&\qquad \qquad \; \, \, set = super .set,& \\
	&\qquad \qquad \; \, \, inc = super .inc,& \\
	&\qquad \qquad \; \, \, backup = \lambda \, \_ : Unit . \, rep.backup \, := \, cons \;  !rep.x \; !rep.backup,& \\
	&\qquad \qquad \; \, \, restore = \lambda \, i : Nat . \, super.set \; (getByIndex \; !rep.backup \; i) \}& \\
	&in& \\
	&\quad let \; newBackupCounter =& \\
	&\qquad \lambda \_ : Unit .& \\
	&\qquad \quad let \; rep = \{ x = ref \; 0, \; backup = ref \; nil [Nat] \} \; in& \\
	&\qquad \quad fix \; (backupCounter \; rep)& \\
\end{flalign*}

\newpage

% Задание 3

\textbf{Задание 3.} Используйте списки, чтобы реализовать класс \verb|AutoBackupCounter|, который автоматически сохраняет состояние при каждом \verb|set|. Реализация \verb|inc| не должна быть переопределена, но должна провоцировать сохранение (\verb|backup|), поскольку метод \verb|inc| в классе \verb|SetCounter| реализован через вызов метода \verb|set|.

\begin{verbatim}
	AutoBackupCounter = {
	    get : Unit -> Nat,
	    set : Nat -> Unit,
	    inc : Unit -> Unit,
	    restore : Nat -> Unit // аргумент - индекс на стеке
	}
\end{verbatim}

\textit{Решение.}

\begin{verbatim}
	AutoBackupCounterRep = { x : Ref Nat, backup : Ref List Nat }
\end{verbatim}

\vspace{-0.5cm}

\begin{flalign*}
	&let \; autoBackupCounter = & \\
	&\quad \lambda \, rep : AutoBackupCounterRep \, .& \\
	&\qquad \lambda \, self : AutoBackupCounter .& \\
	&\qquad \quad let \; super = SetCounter \; rep \; self \; in& \\
	&\qquad \qquad \{ \, get = super .get,& \\
	&\qquad \qquad \; \, \, set = \lambda \, i : Nat . \, (rep.backup := cons \; !rep.x \; !rep.backup; \; super.set \; i),& \\
	&\qquad \qquad \; \, \, inc = super .inc,& \\
	&\qquad \qquad \; \, \, restore = \lambda \, i : Nat . \, super.set \; (getByIndex \; !rep.backup \; i) \}& \\
	&in& \\
	&\quad let \; newAutoBackupCounter =& \\
	&\qquad \lambda \_ : Unit .& \\
	&\qquad \quad let \; rep = \{ x = ref \; 0, \; backup = ref \; nil [Nat] \} \; in& \\
	&\qquad \quad fix \; (autoBackupCounter \; rep)& \\
\end{flalign*}

\end{document}
